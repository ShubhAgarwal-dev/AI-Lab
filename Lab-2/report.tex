\documentclass[a4paper,10pt,reqno,oneside]{amsart}

\usepackage{hyperref}
\usepackage{listings}[language=Python]
\usepackage{amssymb}
\usepackage{amsmath,amsthm}
\usepackage[a4paper, total={6in, 9in}]{geometry}

\hypersetup{pdftitle={AI LAB REPORT 2}}

\begin{document}
\title[Block World Domain (Greedy Approach)]{Lab-2 Report\\Group: 2}
\author{Shubh Agarwal}
\email[Shubh]{210020047@iitdh.ac.in}%
\urladdr{https://shubhagarwal-dev.github.io/}

\author{Saksham Chhimwal}
\email[Saksham]{210010046@iitdh.ac.in}%
\thanks{This paper is in final form}
\date{\today}

\maketitle

\section{Domain Description}

\textbf{State Space}: Our Implementation accepts the states as tuple (x, y, label) for each block in the state. 
For example: (1, 2, C)

\textbf{Start Node and End Node}: Check input.txt for initial node and goal.txt for final node. Below 
is the graphical representation of the initial and final nodes.

\begin{table}[h]
    \begin{tabular}{ccc}
    \textbf{}  & \textbf{}  & \textbf{}  \\
    \textbf{E} & \textbf{}  & \textbf{}  \\
    \textbf{A} & \textbf{C} & \textbf{}  \\
    \textbf{B} & \textbf{D} & \textbf{}
    \end{tabular}
    \caption{Initial State}
\end{table}

\begin{table}[h]
    \begin{tabular}{ccc}
    \textbf{C} & \textbf{} & \textbf{} \\
    \textbf{A} & \textbf{E} & \textbf{} \\
    \textbf{B} & \textbf{D} & \textbf{}
    \end{tabular}
    \caption{Final State}
\end{table}

\textbf{MOVEGEN Algorithm}: We are using 3 \texttt{stack} to find the next generations. First, we will convert 
our state representation into the \texttt{stacks} (x will be stack number, y  will be index of block in that stack and label will 
be the representation), then the top element of non empty stack is moved to other stacks. It gives us all the possible next states. 
Now each possible state is converted back to tuple representation from stack representation. We are using 
\texttt{Hill Climb(\textbf{Greedy})} approach to find the solution using one of the four heuristics:
\begin{itemize}
    \item Manhattam Distance Heuristic
    \item XNOR Heuristic
    \item XNOR-height Heuristic
    \item ASCII-Code Heuristic
\end{itemize}
Pseudocode:
\begin{lstlisting}[frame=single]
    Initialize stack in a 2-D Array from the List of Tuples 
    Initialize List holding all possible States that can be achieved
    FOR EACH Stack:
        Get the TopBlock
        FOR different Stack:
            Add the TopBlock to Pile2
            PUSH this state in Possible State List
            POP the TopBlock from Pile2
        PUSH the TopBlock in Pile
    return the List containing the possible states
\end{lstlisting}

\textbf{GOALTEST Algorithm}: It is just a simple comparision test, simple 
comparing given\_state and goal\_state.\\
Pseudocode:
\begin{lstlisting}[frame=single]
    Get the currentState and the finalState
    If currentState.SORT(key=LABEL) == finalState.SORT(key=LABEL)
        return true
    return false
\end{lstlisting}
\section{Heuristic Functions Considered}
\subsection*{\texttt{xnor$\_$heuristic}}
This heuristic is the one basic heuristic discussed in class. This gives a value \textbf{+1}
to the blocks that are on the correct position w.r.t to the GOALSTATE. And assigns a value of 
\textbf{-1} to the ones that are on the incorrect position w.r.t the GOALSTATE.
\newline
This heuristic has a high possibility of getting stuck and was not able to reach the GOALSTATE many a times.

\subsection*{\texttt{xnor$\_$heuristic$\_$modified}}
This heuristic combines the xnor$\_$heuristic with height.
It works as follows:
\begin{enumerate}
    \item If the item is at the correct position and has a height \texttt{h} then it will assign it a value of \texttt{+h}
    \item If the item is not at the correct position and has a height \texttt{h} then it will assign it a value of \texttt{-h}.
    \item The height starts form 1 at the bottom of the stack. That is the lowest block has a height of \texttt{1}.
\end{enumerate}
\newline
This heuristic was better than the \texttt{xnor$\_$heuristic} and was able to reach the GOALSTATE for some of the inputs.

\subsection*{manhattan$\_$heuristic}
It sums the distance between block in current state and the final state. We return additive inverse of the sum
because our hill climb maximizes the heuristic value.

\subsection*{ascii$\_$heuristic}
It multiplies the height of block by the \texttt{ASCII} value of label and add it if the position of the block 
is correct and substract if the position is incorrect.

\begin{table}[h]
    \begin{tabular}{|c|c|c|}
    \hline
    \textbf{Heuristic} & \textbf{Goal Reached} & \textbf{States Explored} \\ \hline
    \textbf{XNOR}      & Flase                 & 0                        \\ \hline
    \textbf{XNOR\_MOD} & True                  & 3                        \\ \hline
    \textbf{MANHATTAN} & False                 & 1                        \\ \hline
    \textbf{ASCII}     & True                  & 3                        \\ \hline
    \end{tabular}
    \caption{Final Results}
\end{table}



\section{Hill Climbing}
The Hill Climbing approach works as follows:
\begin{enumerate}
    \item It calculates the value of the heuristic of the initial state which is being used currently by the program.
    \item It then calculates the heuristic values for all possible state that can be reached from the current state and stores them in a list.
    \item It chooses the state that has the highest value in the list.
    \item If the chosen value if larger than the heuristic value of the current state then the current state is transformed into the chosen state.
    \item This loop continues till the GOALSTATE is reached or the program halts because of indeterminism.
\end{enumerate}
The program halts under the following conditions:
\begin{enumerate}
    \item If the maximum heuristic value available among the possible states is lower than the current state.
    \item If all the heuristic values are same among the possible states.
\end{enumerate}
\\
\textbf{Time Taken:} We have already shared the number of steps by our heuristics, which is a good rough estimate of
time taken by the heuristics. Time details are shared below:

\begin{table}[h]
    \begin{tabular}{|c|c|l|}
    \hline
    \textbf{Heuristic} & \textbf{Goal Reached} & \textbf{Time Taken(in ms)} \\ \hline
    \textbf{XNOR}      & Flase                 & 0.0036                     \\ \hline
    \textbf{XNOR\_MOD} & True                  & 0.132                      \\ \hline
    \textbf{MANHATTAN} & False                 & 0.00926                    \\ \hline
    \textbf{ASCII}     & True                  & 0.135                      \\ \hline
    \end{tabular}
    \caption{Time Taken}
\end{table}

\textbf{Optimal Solution:}  We are effectively doing restricted \texttt{BFS}, solution if obtained will be optimal
given the heuristics.
\end{document}
