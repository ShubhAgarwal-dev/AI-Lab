\documentclass[a4paper,10pt,reqno,oneside]{amsart}

\usepackage{hyperref}
\usepackage{amssymb}
\usepackage{amsmath,amsthm}
\usepackage[a4paper, total={6in, 9in}]{geometry}

\hypersetup{pdftitle={AI LAB REPORT 2}}

\begin{document}
\title[Block World Domain (Greedy Approach)]{Lab-2 Report\\Group: 2}
\author{Shubh Agarwal}
\email[Shubh]{210020047@iitdh.ac.in}%
\urladdr{https://shubhagarwal-dev.github.io/}

\author{Saksham Chhimwal}
\email[Saksham]{210010046@iitdh.ac.in}%
\thanks{This paper is in final form}
\date{\today}

\maketitle

\section{Domain Description}

\textbf{State Space}: Our Implementation accepts the states as tuple (x, y, label) for each block in the state. 
For example: (1, 2, C)

\textbf{Start Node and End Node}: Check input.txt for initial node and goal.txt for final node. Below 
is the graphical representation of the initial and final nodes.

\begin{table}[h]
    \begin{tabular}{ccc}
    \textbf{}  & \textbf{}  & \textbf{}  \\
    \textbf{F} & \textbf{}  & \textbf{}  \\
    \textbf{B} & \textbf{A} & \textbf{}  \\
    \textbf{E} & \textbf{D} & \textbf{C}
    \end{tabular}
    \caption{Initial State}
\end{table}

\begin{table}[h]
    \begin{tabular}{ccc}
    \textbf{B} & \textbf{C} & \textbf{} \\
    \textbf{D} & \textbf{F} & \textbf{} \\
    \textbf{A} & \textbf{E} & \textbf{}
    \end{tabular}
    \caption{Final State}
\end{table}

\textbf{MOVEGEN Algorithm}: We are using 3 stack to find the next generations. First, we will convert 
our state representation into the \emph{stacks} (x will be stack number, y  will be index of block in that stack and label will 
be the representation), then the top element of non empty stack is moved to other stacks. It gives us all the possible next states. 
Now each possible state is converted back to tuple representation from stack representation. We are using 
\emph{Hill Climb(\textbf{Greedy})} approach to find the solution using one of the four heuristics:
\begin{itemize}
    \item Manhattam Distance Heuristic
    \item XNOR Heuristic
    \item XNOR-height Heuristic
    \item ASCII-Code Heuristic
\end{itemize}

\textbf{GOALTEST Algorithm}: It is just a simple comparision test, simple 
compairing given\_state and goal\_state.

\section{Heuristic Functions Considered}
\section{Hill Climbing}

Saksham you will have to read this,,, If you are a true lober then you will
confess your true lobe for her,,,,

\end{document}
