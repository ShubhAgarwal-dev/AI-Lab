%% Based on a TeXnicCenter-Template by Gyorgy SZEIDL.
%%%%%%%%%%%%%%%%%%%%%%%%%%%%%%%%%%%%%%%%%%%%%%%%%%%%%%%%%%%%%

%------------------------------------------------------------
%
\documentclass[a4paper,12pt,reqno,oneside]{amsart}
%
%----------------------------------------------------------
% This is a sample document for the AMS LaTeX Article Class
% Class options
%        -- Point size:  8pt, 9pt, 10pt (default), 11pt, 12pt
%        -- Paper size:  letterpaper(default), a4paper
%        -- Orientation: portrait(default), landscape
%        -- Print size:  oneside, twoside(default)
%        -- Quality:     final(default), draft
%        -- Title page:  notitlepage, titlepage(default)
%        -- Start chapter on left:
%                        openright(default), openany
%        -- Columns:     onecolumn(default), twocolumn
%        -- Omit extra math features:
%                        nomath
%        -- AMSfonts:    noamsfonts
%        -- PSAMSFonts  (fewer AMSfonts sizes):
%                        psamsfonts
%        -- Equation numbering:
%                        leqno(default), reqno (equation numbers are on the right side)
%        -- Equation centering:
%                        centertags(default), tbtags
%        -- Displayed equations (centered is the default):
%                        fleqn (equations start at the same distance from the right side)
%        -- Electronic journal:
%                        e-only
%------------------------------------------------------------
% For instance the command
%          \documentclass[a4paper,12pt,reqno]{amsart}
% ensures that the paper size is a4, fonts are typeset at the size 12p
% and the equation numbers are on the right side
%
\usepackage{amsmath}%
\usepackage{amsfonts}%
\usepackage{amssymb}%
\usepackage{graphicx}
\usepackage{hyperref}
\usepackage{amssymb}
%------------------------------------------------------------
% Theorem like environments
%
\newtheorem{theorem}{Theorem}
\theoremstyle{plain}
\newtheorem{acknowledgement}{Acknowledgement}
\newtheorem{algorithm}{Algorithm}
\newtheorem{axiom}{Axiom}
\newtheorem{case}{Case}
\newtheorem{claim}{Claim}
\newtheorem{conclusion}{Conclusion}
\newtheorem{condition}{Condition}
\newtheorem{conjecture}{Conjecture}
\newtheorem{corollary}{Corollary}
\newtheorem{criterion}{Criterion}
\newtheorem{definition}{Definition}
\newtheorem{example}{Example}
\newtheorem{exercise}{Exercise}
\newtheorem{lemma}{Lemma}
\newtheorem{notation}{Notation}
\newtheorem{problem}{Problem}
\newtheorem{proposition}{Proposition}
\newtheorem{remark}{Remark}
\newtheorem{solution}{Solution}
\newtheorem{summary}{Summary}
\numberwithin{equation}{section}

\hypersetup{pdftitle={MDS Scribing}}
%--------------------------------------------------------
\begin{document}
\title[Short Title (for the running head)]{Full Title}

\author{Shubh Agarwal}
\email[Shubh]{210020047@iitdh.ac.in}%
\urladdr{https://shubhagarwal-dev.github.io/}

\author{Saksham Chhimwal}
\email[Saksham]{210010046@iitdh.ac.in}%
%\urladdr{http://www.authortwo.uni-atwo.hu}

\author{Abhiram}
\email[A.~Three]{author-three@authorthree-inst.edu}%

\author{Rishikesh}
\email[A.~Three]{author-three@authorthree-inst.edu}%

%\thanks{Thanks for Author One.}
%\thanks{Thanks for Author Two.}
\thanks{This paper is in final form}
\date{\today}
\subjclass{2023 Mathematics for Data Science, CSXXX} %
%\keywords{Keyword one, keyword two etc.}%
%\begin{abstract}
%This is a sample document which shows the most important features of the AMS Journal
%Article class.
%\end{abstract}
\maketitle

\begin{theorem}
$
	\large\textbf{Rank-Nullity Theorem} : 
	$Suppose V is finite-dimensional and $ T \in \mathcal{L} : (V, W).$ Then range T is finite-dimensional and 
	\begin{center}
		$Rank(T)$ + $Nullity(T)$ = $dim(V)$, or
	\end{center}
	\begin{center}
		$dim(Range(T))$ + $dim(Ker(T))$ = $dim(V)$
	\end{center}
\end{theorem}

\begin{proof}
[Proof of the Rank-Nullity Theorem]
Let $B$
\end{proof}
%This text is a sample for a short bibliography. You can cite a book by making use of
%the command \verb"\cite{KarelRektorys}": \cite{KarelRektorys}. Papers can be cited
%similarly: \cite{Bertoti97}. If you want multiple citations to appear in a single set
%of square brackets you must type all of the citation keys inside a single citation,
%separating each with a comma. Here is an example: \cite{Bertoti97, Szeidl2001,
%Carlson67}.

%\begin{thebibliography}{9}                                                                                                %
%\bibitem {KarelRektorys}Rektorys, K., \textit{Variational methods in Mathematics,
%Science and Engineering}, D. Reidel Publishing Company,
%Dordrecht-Hollanf/Boston-U.S.A., 2th edition, 1975

%\bibitem {Bertoti97} \textsc{Bert\'{o}ti, E.}:\ \textit{On mixed variational formulation
%of linear elasticity using nonsymmetric stresses and displacements}, International
%Journal for Numerical Methods in Engineering., \textbf{42}, (1997), 561-578.

%\bibitem {Szeidl2001} \textsc{Szeidl, G.}:\ \textit{Boundary integral equations for
%plane problems in terms of stress functions of order one}, Journal of Computational and
%Applied Mechanics, \textbf{2}(2), (2001), 237-261.

%\bibitem {Carlson67}  \textsc{Carlson D. E.}:\ \textit{On G\"{u}nther's stress functions
%for couple stresses}, Quart. Appl. Math., \textbf{25}, (1967), 139-146.
%\end{thebibliography}
\end{document}
